\documentclass[12pt,letterpaper]{book}
\usepackage[utf8]{inputenc}
\usepackage{amsmath}
\usepackage{amsfonts}
\usepackage{amssymb}
\usepackage{graphicx}
\title{Introduction}
\author{RaylaVR}

\begin{document}
\pagenumbering{arabic}


\chapter{Introduction}

\section{What is Marching Percussion?}

\section{The Benefits of Marching Percussion}

\section{Material Requirements}

\paragraph{What you need}
\begin{itemize}
    \item Practice Pad and Sticks
    \item Some form of communications device/internet access (basic smartphone)
    \item A metronome
    \item A folder/binder for music
    \item Consistent transportation
    \item A whole lot of food and water
    \item A good attitude!
\end{itemize}

\paragraph{Other helpful tools}
\begin{itemize}
    \item A Nice Practice Pad
    \item A Tenor Pad (for tenor players)
    \item Heavy Practice Sticks
    \item Music stand/drum stand
\end{itemize}


\section{Teaching Methodology and Philosophy}

This guide is written to be taught like a typical classroom course and will be taught with a similar structure, however it is designed to minimize needless work and give effective practice assignments and processes.

\section{Communications and Scheduling}

It is of the upmost importance that both Parent/Guardian(s) and Students be in contact with the instructor over the season as being present is critical.  As a group activity, missing rehearsal is not only detrimental to one's self but also to the group.  Scheduling information will be provided as quickly and diligently as possible, and in return it is expected that absences are accounted for or known about in advance.  

\section{Physical and Mental Requirements}

\section{Social Expextations and Culture}

\section{Get Inspired!}

\section{Student and Parental Agreements with the instructor}

\section{Grading}

\paragraph{Breakdown}

Participation: 30\%
Attendance: 50\%
Music Tests: 20\%

\end{document}