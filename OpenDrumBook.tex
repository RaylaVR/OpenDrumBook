\documentclass[12pt,letterpaper]{book}
\usepackage[utf8]{inputenc}
\usepackage{amsmath}
\usepackage{amsfonts}
\usepackage{amssymb}
\usepackage{graphicx}
\title{Open Drum Book}
\author{RaylaVR}

\begin{document}
\pagenumbering{gobble}
\maketitle
\newpage
\pagenumbering{arabic}
\tableofcontents







\part{Preliminary}


\chapter{Forward}
% This will introduce the book itself, Its purpose, how to use it ect.
% Why I made the book, and personal message to its reader

\chapter{Teacher's Usage Guide}
% This instructs the teacher on how exactly to utilize this book and 
% other avalible resources. It talks about teaching philosophy, best practices
% common teaching mistakes and other essential information for teaching out of this book.

\chapter{Marching Technique}

\chapter{Useful Preliminary Documents}

\section{What is Marching Percussion?}

\section{The Benefits of Marching Percussion}

\section{Material Requirements}

\paragraph{What you need}
\begin{itemize}
    \item Practice Pad and Sticks
    \item Some form of communications device/internet access (basic smartphone)
    \item A metronome
    \item A folder/binder for music
    \item Consistent transportation
    \item A whole lot of food and water
    \item A good attitude!
\end{itemize}

\paragraph{Other helpful tools}
\begin{itemize}
    \item A Nice Practice Pad
    \item A Tenor Pad (for tenor players)
    \item Heavy Practice Sticks
    \item Music stand/drum stand
\end{itemize}


\section{Teaching Methodology and Philosophy}

This guide is written to be taught like a typical classroom course and will be taught with a similar structure, however it is designed to minimize needless work and give effective practice assignments and processes.

\section{Communications and Scheduling}

It is of the upmost importance that both Parent/Guardian(s) and Students be in contact with the instructor over the season as being present is critical.  As a group activity, missing rehearsal is not only detrimental to one's self but also to the group.  Scheduling information will be provided as quickly and diligently as possible, and in return it is expected that absences are accounted for or known about in advance.  

\section{Physical and Mental Requirements}

\section{Social Expextations and Culture}

\section{Get Inspired!}

\section{Student and Parental Agreements with the instructor}

\section{Grading}

\paragraph{Breakdown}
Participation: 30\%\\
Attendance: 50\%\\
Music Tests: 20\%\\











\part{Beginner}
\chapter{Introduction and Overview}
\section{Introduction}

This set of lessons is designed to take a someone with no previous music experience up to being able to play what is required for easy to intermediate marching bass drum music.  It serves as the foundation for all marching percussion and is a prerequisite for all future levels.  It is recommended for middle school and early high school students.

The Beginner Course is divided into two chapters which each have a relevant exercise at the end for students to apply their skills.  These are written to be somewhat challenging and serve as good exercises to play in your warm-up sequence.


\chapter{Gotta Start Somewhere}

\paragraph{The Basics of Technique, Music Reading, and Subdivision}

In this chapter, we will learn the absolute bear minimum of what it takes to play drums.  The focus will be on reading music and playing to a tempo, syncing your feet with your playing in a \textit{marktime}, and understanding the 4 stroke types.  It will be slow, frustrating, and admitedly pretty boring.  These barebones skills are things that you will do everyday as they form the fundemental components of marching percussion.
\end{document}