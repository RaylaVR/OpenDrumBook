\documentclass[12pt,letterpaper]{book}
\usepackage[utf8]{inputenc}
\usepackage{amsmath}
\usepackage{amsfonts}
\usepackage{amssymb}
\usepackage{graphicx}
\usepackage{indentfirst}
\title{Open Drum Book}
\author{RaylaVR}

\begin{document}
\pagenumbering{gobble}
\maketitle
\newpage
\pagenumbering{arabic}
\tableofcontents







\part{Preliminary}


\chapter{Forward}
% This will introduce the book itself, Its purpose, how to use it ect.
% Why I made the book, and personal message to its reader

\chapter{Teacher's Usage Guide}
% This instructs the teacher on how exactly to utilize this book and 
% other avalible resources. It talks about teaching philosophy, best practices
% common teaching mistakes and other essential information for teaching out of this book.

\chapter{Marching Technique}

\chapter{Useful Preliminary Documents}

\section{What is Marching Percussion?}

\section{The Benefits of Marching Percussion}

\section{Material Requirements}

\paragraph{What you need}
\begin{itemize}
    \item Practice Pad and Sticks
    \item Some form of communications device/internet access (basic smartphone)
    \item A metronome
    \item A folder/binder for music
    \item Consistent transportation
    \item A whole lot of food and water
    \item A good attitude!
\end{itemize}

\paragraph{Other helpful tools}
\begin{itemize}
    \item A Nice Practice Pad
    \item A Tenor Pad (for tenor players)
    \item Heavy Practice Sticks
    \item Music stand/drum stand
\end{itemize}


\section{Teaching Methodology and Philosophy}

This guide is written to be taught like a typical classroom course and will be taught with a similar structure, however it is designed to minimize needless work and give effective practice assignments and processes.

\section{Communications and Scheduling}

It is of the upmost importance that both Parent/Guardian(s) and Students be in contact with the instructor over the season as being present is critical.  As a group activity, missing rehearsal is not only detrimental to one's self but also to the group.  Scheduling information will be provided as quickly and diligently as possible, and in return it is expected that absences are accounted for or known about in advance.  

\section{Physical and Mental Requirements}

\section{Social Expextations and Culture}

\section{Get Inspired!}

\section{Student and Parental Agreements with the instructor}

\section{Grading}

\paragraph{Breakdown}
Participation: 30\%\\
Attendance: 50\%\\
Music Tests: 20\%\\





\part{Beginner}
\chapter{Introduction and Overview}
\section{Introduction}

This set of lessons is designed to take a someone with no previous music experience up to being able to play what is required for easy to intermediate marching bass drum music.  It serves as the foundation for all marching percussion and is a prerequisite for all future levels.  It is recommended for middle school and early high school students.

The Beginner Course is divided into two chapters which each have a relevant exercise at the end for students to apply their skills.  These are written to be somewhat challenging and serve as good exercises to play in your warm-up sequence.


\chapter{Gotta Start Somewhere}

\paragraph{The Basics of Technique, Music Reading, and Subdivision}

In this chapter, we will learn the absolute bear minimum of what it takes to play drums.  The focus will be on reading music and playing to a tempo, syncing your feet with your playing in a \textit{marktime}, and understanding the 4 stroke types.  It will be slow, frustrating, and admitedly pretty boring.  These barebones skills are things that you will do everyday as they form the fundemental components of marching percussion.

\section{Section 1: Hello Drums!}

\subsection{Introduction}

% This should introduce what marching drums are and what will be learned in this lesson.

\subsection{Marking Time and The Metronome}

\paragraph{What is time?} 

Before we can even think about playing the drums, we need to learn how to keep time. The way we do this in marching band is with our feet.  The synchronised movement not only looks cool, but also makes it easy to know where we are in the music and how fast it is.  To practice this, we are going to learn how to *Mark Time*.

\textit{Marking Time} is what we do to simulate marching when standing still.  The idiom "to mark time" is actually fitting to what we use it for.

\subparagraph{}To idly wait; to do nothing except observe the passage of time.

In its most basic form Marking Time is just walking in place to a tempo; you pick your foot up and put it back down again in time with the music.  We are observing the passage of musical time and marking it with a foot placement.  If we can all do this together, then we are observing the passage of musical time together.  

So how do we know what the tempo is?  We need a frame of refrence, and this is where the metronome comes in.

\paragraph{Let's Try It!}

Set the metronome to 120 bpm (beats per minute).  Something you may notice about this specific tempo is that it is exactly twice per second.  And that makes sense as a minute contains 60 seconds.  120 is double 60, therefore for every second there will be one musical count.  This specific tempo highlights that tempo and keeping time is all about *fractions*, which is something to keep in mind for later.

We all hear the clicking at a regular interval.  It is perfectly regular in fact, which makes it an excellent set of training wheels for learning to play drums.  Clap along with me to the metronome.

% Have your students clap along to this clicking and try to get as close to the click as possible.

Not that hard right?  This time we will try again, however part way through, we'll mute the metronome.  Keep clapping and see if you can maintain that very same clapping speed.  

% Have your students clap along to the click track then without pausing it mute the metronome.  After you feel that tempo has strayed sufficiently, unmute the metronome and let your students hear how far off they were.

Not so easy now is it?  This is why we practice with a metronome, to teach ourselves how to maintain a tempo.  This is also why we mark time!  It will act as our metronome on the marching feild as it is almost always in sync with our music's written tempo.  Lets try marking time to the metronome!

% This is where you can optionally teach whatever your desired mark time is.  In this guide, we will use a very simple flat-footed approach.  This approach was chosen for its ease of learning, clear definition of time, and lack of obsqurity when a student makes an error.

The technique to mark time is very simple, but must be defined so that we all do it the same way.  Standing with our feet together pick up your left foot to your ankle bone, then softly stomp your foot back onto the ground flat.  The lift of the foot should be just before you put it back down again.

\subsection{Musical Prerequisites}

% TODO: Teaches the basics of music notation and presents a few examples of how one would go about reading music.

This simple act of playing to a tempo is the very core of drums. Now that we have established Mark Time, we can start applying it to some real music.  Figure 1 shows is going to be our first peice of music.  It is very short, one measure long with only one note.  Anyone who knows how to read music would chuckle at this but we can learn a whole lot about marching percussion by defining this.

Take a look at Figure 1a. 

\subsection{Playing the Drum for the first time}

% TODO: Goes over the basic mechanics of a drum stroke, the grip, and the axies of motion. Wrist-arm-arm-wrist, up-down, repetitive strokes. Do a counted number of strokes then stop.

\subsection{8-8-16}

% TODO: Have them read and play 8-8-16, add the feet and make sure everyone is lining it up.

\subsection{8th Note Reading Exercises}

% TODO: 8th note reading/timing exercises while marking time

\subsection{Quiz:}

% Quiz them on reading skill and 8-8-16 looking at their technique to make sure that they are using about the right amount of arm/wrist, aren't massivly over/undersqueezing the stick, and using the fulcrum to bounce the stroke. Don't be overly critical with technique as it will be improved upon quickly over the next few lessons.









\end{document}